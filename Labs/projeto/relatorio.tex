\documentclass[12pt, letterpaper]{article}
\usepackage[utf8]{inputenc}

\usepackage{amsmath}
\usepackage{indentfirst}
\usepackage{hyperref}
\hypersetup{
    colorlinks=true,
    linkcolor=black,  
    urlcolor=cyan,
    }

\title{Projeto - MC404}
\author{Rian Radeck Santos Costa - 187793 \\ Cirilo Max Macêdo de Morais Filho - 168838}
\date{30 de Junho de 2022}
\renewcommand*\contentsname{Sumário}

\begin{document}

\maketitle
\newpage
\tableofcontents
\newpage

\section{Esclarecimentos}
	Todas as operações e conversões foram preparadas para inteiros sinalizados de 32 bits.

	Nós decidimos adicionar uma operação de exponenciação pios achamos que poderia agregar positivamente no conjunto do trabalho.

	O projeto foi dividido em duas partes:
	\begin{itemize}
		\item{Operações  (projeto1.s)}
		\item{Conversões (projeto2.s)}
	\end{itemize}
	essa divisão foi feita pois o simulador RISC-V utilizado não aceita códigos muito extensos.

	Definições:
	\begin{itemize}
		\item{$n$ é a quantidade de bits de $a_0$.}
		\item{$m$ é a quantidade de bits de $a_1$.}
	\end{itemize}

\newpage
\section{Operações}
	Estas operações estão no arquivo projeto1.s, elas foram feitas baseadas com o conhecimento adiquirido dentro de sala e com base no material fornecido pelo professor Ricardo Pannain no classroom.
	\subsection{Soma}
		A operação feita no progrma é $a_0 \leftarrow a_0 + a_1$.

		O overflow foi verificado somando os valores absolutos dos registradores e caso os seus sinais fossem iguais, o resultado não poderia ser negativo.

		Complexidade: $O(1)$
	\subsection{Subtração}
		A operação feita no progrma é $a_0 \leftarrow a_0 - a_1$.

		A execução dessa operação é feita invertendo o sinal do registrador $a_1$ e chamando a função soma.

		Complexidade: $O(1)$
	\subsection{Multiplicação}
		A operação feita no programa é $a_0 \leftarrow a_0 \times a_1$.

		O algoritmo utilizado foi o algoritmo de Booth, sendo assim o tratamento de sinal era feito no fim do algoritmo, após a multiplicação dos fatores. O overflow foi verificado em cada parte da soma das parcelas da multiplicação, assim se em algum momento o produto se tornasse negativo havia ocorrido o overflow.

		Complexidade: $O(n)$.
	\subsection{Divisão}
		A operação feita no programa é $a_0 \leftarrow \left\lfloor \frac{a_0}{a_1} \right\rfloor$ e $a_1 \leftarrow a_0 \mod a_1$ 

		O algoritmo utilizado foi o seguinte:
		\begin{itemize}
			\item{Verifica se o divisor é 0.}
			\item{Se o divisor for menor ou igual ou dividendo.}
			\begin{itemize}
				\item{Acrescenta 1 no quociente.}
				\item{Subtrai o divisor dos bits considerados até o memento do dividendo.}
			\end{itemize}
			\item{Caso contrário}
			\begin{itemize}
				\item{Acrescenta 0 no quociente.}
				\item{É considerado o próximo bit mais significativo do dividendo}
			\end{itemize}
			\item{Quando não houver mais bits para se considerar do dividendo, o que sobrou do dividendo é o resto}
		\end{itemize}

		Complexidade: $O(n)$
	\subsection{Exponenciação}
		A operação feita no programa é $a_0 \leftarrow {a_0}^{a_1}$

		O algoritmo utilizado foi o de exponenciação rápida. Ele funciona da seguinte maneira:
		\begin{itemize}
			\item{Caso o expoente seja 0, o resultado é 1}
			\item{Caso o expoente seja par, dividimos por 2 e calculamos o seu quadrado com uma multiplicação}
			\item{Caso o expoente seja ímpar, subtraimos 1 do expoente e calculamos o caso par, depois multiplicamos pela base.}
		\end{itemize}
		Complexidade: $O(n\log_2{m})$

\newpage
\section{Conversões}
	\seubsection{Algoritmo}
		Sabemos que um número é representado em certa base da seguinte maneira:
		$$
		N = c_0 \times b^0 + c_1 \times b^1 + c_2 \times b^2 + \cdots
		$$
		onde $b$ é a base e $c_0$ é seu algarismo menos significativo e $0\leq c_k < b$ $\forall$  $k$ $\in \mathbb{Z}$.

		O algoritmo usado na conversão de bases foi o seguinte:
		\begin{itemize}
			\item{Tome um expoente $e$ grande o suficiente $(N \leq b^e)$.}
			\item{Tome o seu algarismo $c$ como $b - 1$.}
			\item{Diminua seu algarismo até $c \times b^e \leq N$.}
			\item{Acrescente esse algarismo na resposta.}
			\item{Diminua $N$ de $c \times b^e$.}
			\item{Subtraia 1 do expoente}
			\item{Se o expoente é negativo, o algoritmo para.}
		\end{itemize}
	Todas as conversões foram feitas utilizando esse algoritmo.

	\subsection{Binário}
		\begin{itemize}
			\item{Leitura: string.}
			\item{Escrita: string.}
		\end{itemize}
	\subsection{Hexadecimal}
		\begin{itemize}
			\item{Leitura: string.}
			\item{Escrita: cadeia de caracteres.}
		\end{itemize}

	\subsection{Decimal}
		\begin{itemize}
			\item{Leitura: RISC-V.}
			\item{Escrita: RISC-V.}
		\end{itemize}
\end{document}